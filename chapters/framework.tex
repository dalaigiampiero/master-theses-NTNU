\chapter{Framework}
\label{chap:framework}

In this section, the information from the previously discussed literature is compiled into a tentative framework for early ASD detection. The framework consist of a set of eye parameters measurements and test procedures, which eventually yield specific objectively measured outcome variables, which are needed in order to assess the reliability and the validity of oculomotor function measurements (Tab.~\ref{tab:frameworksummary}). The review article from \cite{smyrnis2008guidelines} strives to provide guidelines for the standardization of oculomotor function tests in psychiatric research, and it is used as a canvas upon which the framework is built on. 

Literature describes several established methods for eliciting eye movement responses in ASDg (reported for example by \citealp{johnson2016review,brenner2007visualsearch,zalla2016saccades,falck-ytter2013eyetrackingASD,papagiannopoulou2014review}, etc), like visual search tasks, embedded figures test, visually-guided saccade tasks and step-gap-overlap paradigms, paired visual preference paradigm, semi-naturalistic and naturalistic settings, etc, and each method demands specific types of stimuli materials. The following framework keeps into account only the methods which specifically elicit the eye parameters selected in Section~\ref{sec:eyeparameters} as most probable indicators for early ASD detection, and which can be applied to the target age group 12-24 month.

The experimental setting and procedure protocols are just drafted in the framework. This is due to the fact that the present study serves more as pilot test which is not meant to provide standardized and replicable guidelines at the current stage. As explained in Section 5. Results and Section 6. Discussion, the framework seems to be applicable to the target children, but there is still a noticeable amount or research to do before providing standardized procedure for ASD diagnostics with eye trackers, including experiment setup, stimuli, measurement calculations and outputs.
It is worth remembering that at this stage the framework is based completely on findings in literature, therefore it can show only theoretical, face, content and construct validity, as explained in Section~\ref{sec:fwkreliabilityvalidity}.

\begin{table}[tbp]
  \centering
  \begin{tabular}{c|c}
    Age  & IQ  \\ 
    \hline
    10   & 100 \\
    20   & 100 \\
    30   & 150
  \end{tabular}
  \caption{Framework executive summary}
  \label{tab:frameworksummary}
\end{table}


\section{Subjects}
\label{sec:fwksubjects}

Given that the period in which clinicians can formulate an ASD diagnosis by using behavioral assessment is when the children are around 24 and 32 months old (Minchiello, personal communication), providing information which could lead to an earlier detection of ASD in an earlier stage than the usual one is a key point of an eye tracking framework. Children in the age group of 12-24 months seem suitable candidates for the framework.

Indeed, as \cite{towie2016screening} explain, different timing patterns of symptom emergence among ASDg make the ages before 12 months to 18 months particularly ambiguous. Phenotypic variability (quantity and quality of visible symptoms) and the variable degree of overall severity are a known feature of ASD, and young children with milder symptoms are more difficult to diagnose under 24 months of age. Moreover, the observable behavioral ASD markers become more evident from about 12 months on, and they can be interpreted either as the result of earlier divergent neurocognitive processes and also a divergent foundation that may produce further ASD symptoms. As explained in Section~\ref{sec:eyetrackingdiagnosis}, the eye tracking framework is meant to be used together with the already validated behavioral diagnostic praxis. Therefore some divergent behavioral patterns in children should be noticeable by the clinicians in order to collect qualitative information along with the quantitative eye tracking measurements.

Therefore there is a need for more precise early detection instruments for ASD particularly below the 24 months of age threshold, which is the period of life of the children in which it is more difficult to spot subtle signs and symptoms, and above 12 months of age, moment from which some symptoms are visible and that can raise suspect of ASD insurgency and provide qualitative assessments.

\section{Apparatus}
\label{sec:fwkapparatus}

Summarizing from \cite{smyrnis2008guidelines}, the apparatus for measuring eye movements available to the researchers nowadays are electrooculography (EOG), infrared limbus or pupil detection, video camera capture with pattern recognition algorithms, and contact lenses with search coil.
Between all the apparatus, the one which seems the most viable for testing with small children is the infrared pupil reflection one, in which the infrared light emitted by the device illuminates the eyes and the reflection from within the pupil is detected by photodetectors. A video camera capture of eye movements with pattern recognition algorithms of the pupil is also a good option, even if low sampling frequency can be an issue. Using eye tracking systems with sampling frequencies below 100 Hz is detrimental to the quality of results.
EOG needs skin electrodes placed around the eye and it is prone to artifacts from the movements of the muscles in the face area (which the infrared apparatus do not show). The contact lenses with search coil are too invasive and can cause distress.

\cite{sasson2012children} provide recommendations on the usage of eye trackers on small children with autism. They advice to avoid head stabilization (e.g. chin-rest or head mounted systems), preventing the children to oppose resistance. Remote eye trackers are preferable, since they also allow for minor head movements. Eye trackers integrated within the display monitor or table-top are preferable, since relatively unobtrusive. Sampling rates of 50 Hz are useful only in visual scanning tasks, while sampling rates of \(\geq\)250 Hz are necessary for investigating more subtle oculomotor behaviors like smooth pursuit and saccades. 

\section{Eye tracking parameters}
\label{sec:fwkparameters}

Summarizing from the results and the reflections from the literature included in Section~\ref{sec:eyeparameters}it seems that so far the most reliable parameters which could discern ASDg from TDg, and which could potentially be helpful in supporting the early diagnosis of ASD in small children are:
\begin{enumerate}
    \item Open-loop smooth pursuit gain;
    \item Closed-loop smooth pursuit gain;
    \item Standard deviation of saccade gain.
\end{enumerate}


\import{chapters/}{frameworkMethods.tex}

\import{chapters/}{frameworkDataAnalysis.tex}



\section{Experiment setting and testing procedure}
\label{sec:fwksettingprocedure}

\cite{sasson2012children} provide practical guidelines for setting up the experimental environment and the testing procedure for eye tracking studies involving young children with autism, which are also suitable for small children who has not been diagnosed yet. The guidelines are summarized as follows.
\begin{enumerate}
    \item Sober furnishing and dim lightning help to keep the children’s attention on the screen. It is better to avoid to test in a completely darkened room or using loud sounds to avoid to make the children uncomfortable. The experimenters should position themselves out of the participant’s view or put a partition between themselves and the eye-tracking station, while still keeping a view on the participant.
    \item Children should be accompanied by a caregiver throughout the testing session and the researchers should wait for the child to be acquainted and comfortable with the research environment. Having children's videos or cartoons playing on the display monitor can help with this task, and also it gives the opportunity to position the equipment at the right distance from the child and proceed with the calibration.
    \item Especially when the children are very small (and they require body support) they can sit on the caregiver’s lap, ensuring that only the children’s eyes are captured by the eye tracker.
    \item Calibration should use animated stimuli accompanied by sound, which should be more effective in capturing the children’s attention. A 5-point sequence should be brief enough to retain the child's attention while also providing an accurate calibration.
    \item The visual stimuli and the task should be concise and compelling enough to maximize the children’s attention, requiring minimal effort (e.g. passive viewing tasks). Adding inter-stimulus animated materials can help to recover the children’s interest and keep them entertained. Re-calibration should be administered between tasks, checking and correcting possible calibration drift (3 degrees of visual angle) 
\end{enumerate}

These guidelines are easier to follow in a controlled laboratory, however it is not impossible to meet these requirements also in a home environment, which can be suitable for making the child and the caregiver feel more at ease during the whole procedure and therefore providing more chances to collect more complete measurements. Remote eye trackers, especially if mounted on a laptop computer acting as a portable measuring station, can be quite resilient to environment changes and subject positioning if the ambient lightning is suitable.



\import{chapters/}{frameworkValidity.tex}

