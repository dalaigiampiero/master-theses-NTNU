\chapter{Background}
\label{chap:background}

The background of the current study embraces two interdisciplinary areas, namely the clinical praxis for the assessment of ASD (Section~\ref{sec:clinicalassessment}), belonging to the healthcare and psychology research fields, and the measurement of specific eye parameters with eye tracking technologies (Section~\ref{sec:eyeparameters}), belonging more to the human factors and HCI research fields. The eye tracking framework presented in the next chapter (\ref{chap:framework}) builds upon and tries to make a synthesis of these two research areas.
Some of the information in this section were collected by the author during the literature review for the IMT4898 Specialisation in Interaction Design course and the interview study for IMT4215 Specialization Project, in the autumn of 2017. Therefore some concepts can be found in those previous reports.
The essential literature review (initially started during the IMT4898 course) retrieved a total amount of around 160 relevant papers. The researcher narrowed down to the amount of papers reported in the reference list of this thesis by following a traditional approach \citep[pp. 74-76]{jesson2011literaturereview} with a focus on the meta-analysis of eye parameters, user sample characteristics (primarily age and diagnosis), methods, stimuli material, experimental procedure and outcomes. No systematic approaches were practiced, therefore no claims of scientific evidence are made on the literature. The review provides only the researcher's point of view on the state of the art of the research field.


\import{chapters/}{clinicalAssessment.tex}

\import{chapters/}{eyeParameters.tex}




