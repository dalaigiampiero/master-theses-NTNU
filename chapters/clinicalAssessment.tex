\section{Clinical assessment of ASD: current setting, problems and perspectives}
\label{sec:clinicalassessment}

In order to understand the context in which eye tracking technologies might find their application, in this section some notions about the clinical assessment of ASD are reported, including the current practices, tests and problems. Clinicians already assess the children’s gaze behavior as an important index of risk related to ASD, and they already collect a series of behavioral information through the administration of specific tests which provide a rich qualitative description of the children’s status. Eye trackers can provide a further quantitative measurement tool which should be used together with the current diagnostic praxis.


\subsection{Clinical practice}
\label{sec:clinicalpraxis}


Current clinical practices base the ASD diagnostic procedure on a variety of well validated and specific behavioral screening and diagnostic tests, which are described in detail by \cite{towie2016screening}, \cite{maganto2017screening} and \cite{charman2013measuerement}. As these articles further describe, these instruments can be used, and seem to start to be most useful, during the second year of the children (12 months of age to 24, up to 36 for some tests). However, as highlighted by an interview with Simone Minichiello\footnote{Simone Minichiello is a logopedist expert in clinical diagnosis and rehabilitation of children with neuropsychiatric disorders (among which ASD). He currently works at the Centro Ferrarese di Neuropsichiatria, Neuropsicologia e Riabilitazione - Piccolo Principe (Ferrarese Center for Neuropsychiatry, Neuropsychology and Rehabilitation - The Little Prince), in Ferrara (FE), Italy. The full transcript of the interview is available in the IMT4215 Specialization Project (autumn 2017) report written by the author of this study.}(Minichiello, S., personal communication, November 12, 2017), 24 to 32 months of age is the period in which more children are brought to specialistic structures for receiving a diagnosis of neuropsychiatric disorders. Before 22-24 months of age is quite rare to find children in this kind of diagnosis and rehabilitation structures. Therefore, from a clinical perspective and with the current methods, the early diagnosis happens around 22 and 24 months of age. Before this age threshold, the early symptoms of ASD (especially the ones related with social interaction) can be too subtle both for parents and for the general practitioners, and also the screening tests might not be sensitive enough to subtle signs of early ASD. After 33-36 months it is easier to diagnose ASD since the symptoms are already pervasive, clear and evident. The more the children is young and ASD symptoms are not severe, the more complex is the diagnosis process.\\
From the same interview emerged that diagnosing ASD in the first 24-30 months of life of a child poses particular challenges to clinicians, for example:
\begin{itemize}
    \item Behaviors related particularly to social interaction, the usage of language and how the child perceive and act towards the world become more evident and clear only when the children are a bit older (after the 24th month of life \citep{orlandi2014advancedtools,bocchi2012earlydiagnosis}). Indeed, as also \cite{vargas2016diagnosis} describe, problems related to early detection of ASD are the difficulty for the parents to detect subtle symptoms like those that characterize ASD in early stages (especially in the first child), the lack of information from pediatricians and other professionals to detect ASD early, and the assessment tools currently in use by specialists sometimes are inadequate (too specific or not specific enough, as Minichiello describes).
    \item Comorbidity is relatively frequent, meaning that other neurodevelopmental problems are present along with ASD.
    \item Currently there are not instrumental diagnostics for ASD (e.g. no blood sampling, EEG or MRI which can detect ASD). As also \cite{samad2017markers} describe, due to the fact that currently there are no diagnostic biomarkers for ASD, the disorder is commonly identified through direct visual observation of atypical behaviors. These methods have limitations in identifying subtle behavioral traits which may gradually lead to more complex behavioral impairments over the developmental period of the child with ASD.
    \item There are not strict procedures for ASD diagnosis. Clinical procedures are aimed to describe the symptomatology and at the same time intervene with therapy, waiting for providing a prognosis until the symptoms are clearer.
\end{itemize}

As Minichiello further explains, the result of these issues is that it is difficult to asses the ASD symptom severity in this particular moment of the children lifecycle. Only if the ASD behavioral symptoms are very severe, it is easier to formulate a diagnosis. Due to the complexity of ASD and that the diagnostic behavioral tests are prone to subjective interpretation, a team of specialized clinicians (e.g. child neuropsychiatrist, developmental psychologist, logopedist, neuro-psychomotrist) need to review together the information and the tests about the child, in order to reduce the risk of incorrect diagnosis.
Great benefit could come from providing instrumental and objective assessment tools (and relative procedures) developed specifically for this age group, which can assist the current diagnostic practices. As \cite{bolte2016detection} discuss, objective technology-based techniques of research into early ASD have potential to reveal unknown subtle atypicalities in the developmental processes leading to ASD, and to be used in paediatric and psychiatric clinical practice. This is because these technologies seek to identify biological markers of ASD for earlier diagnosis and biologically defined treatment goals.


\subsection{Current ASD screening and diagnostic instruments}
\label{sec:screeningdiagnostics}

As \citep{bolte2016detection} summarize, there are two main groups of methods for assessing the development in the first months and years of life of children, as well as responses to interventions:
\begin{enumerate}
    \item Informant and clinician-based behavioural methods (e.g questionnaires, observation scales, interviews developmental tests), which are observational, subjective and sometimes qualitative.
    \item Technology-based and/or measurements of basic cognitive or neurological processes and structures (e.g. eye tracking, electroencephalography (EEG), event-related potentials (ERPs), magnetic resonance imaging (MRI), positron emission tomography (PET), transcranial magnetic stimulation (TMS), retrospective video analysis, preferential looking experiments, etc), which are direct, objective and mostly quantitative. These methods show potential to reveal previously unknown subtle atypicalities in the developmental processes leading to ASD, useful for paediatric and child and adolescent psychiatric clinical practice.
\end{enumerate}
The current clinical practice is based on the first group of methods, the observational screening and diagnostic tests, some of which are considered the gold-standards. Some tests are aimed to detect a series of signs of high probability of ASD, then others are carried out in series in order to rule out symptoms which are shared with other conditions and to reach a clinical diagnosis \citep{vargas2016diagnosis}.\\
As \cite{maganto2017screening} explain, over the last two decades more than 20 screening instruments aimed at prospectively identifying children with ASD have been developed and made available internationally, even though these instruments are not yet at an optimal level. The major issue is the difficulty in determining which constructs (e.g. language delays, repetitive behaviours, social interaction, interest in others, joint attention, etc) are essential for discerning between children who are at risk for ASD, at risk for other developmental disorders and typically developing children. Therefore, if construct validity is at debate, practitioners and researchers cannot tell if the different screening tools are measuring the same behaviors. In ASD, the major construct which score higher seem to be social interaction, interest in others, joint attention and atypical eye contact, while motor abnormalities cannot discern clearly between ASDg and children with other other developmental disorder. Moreover, also the content validity of these instruments is largely related to the opinions of the people performing the validation. In order to assess the efficacy and utility of the screening instruments, the following parameters are computed \citep{charman2013measuerement}: (1) sensitivity (the proportion of individuals with a disorder who have a positive screen result), (2) specificity (the proportion of individuals with a disorder who have a negative screen result), (3) positive predictive value (the proportion of individuals with a positive screen result who have the disorder).\\
\cite{towie2016screening} illustrate how in the age range between 18–24 months and 30–36 months there are two levels of \textbf{screening instruments}, which aim is to identify children in need of further monitoring or diagnostic evaluation:
\begin{itemize}
    \item Level 1: They are intended to screen at a population level (i.e all children regardless of their risk level for developmental disabilities). Examples of these screening instruments are the Modified Checklist for Autism in Toddlers (M-CHAT), Infant-Toddler Checklist (ITC) and First-Year Inventory (FYI).
    \item Level 2: They are applied to children at risk (showing signs of ASD or another type of delay or disability, detected by the parents or pediatrician), in order to route the children to more thorough evaluations. Examples of these screening instruments are the Screening Test for Autism in Two-Year-Olds (STAT), Parent Observation of Early Milestones Scale (POEMS) and Autism Detection in Early Childhood (ADEC).
\end{itemize}
\cite{charman2013measuerement} describe the \textbf{diagnostic instruments} in use for ASD, which are used to structure the information-gathering from both parents and identified children within a diagnostic assessment. Some examples are Autism Diagnostic Observation Schedule (ADOS, a semi-structured, standardised observation of children and adults interacting during a series of protocols of activities, useful from 12 months and up; \citealp{lord2000ADOS}), Autism Diagnostic Interview-Revised (ADI-R, a standardised semi-structured interview useful from 12 months and up; \citealp{rutter2003ADI-R}) Diagnostic Interview for Social and Communication Disorders (DISCO, useful at all ages) and Developmental, Dimensional and Diagnostic interview (3Di, useful on unselected clinical and general population samples).
Both the screening and diagnostic instruments enlisted are often limited in their power to correctly discern ASDg and TDg, especially for marginal cases. Even though some of the diagnostic tests have strong predictive validity (e.g. ADOS and ADI-R if used together), clinicians do not rely solely on these assessments in order to formulate a diagnosis. Experienced clinical judgment and training with the instruments still impact on the accuracy of the diagnosis, regardless the valuable standardization provided by the tests. 


\subsection{Where eye tracking could fit in the screening and diagnostic process}
\label{sec:eyetrackingdiagnosis}
Given the clinical practice framework illustrated in the previous section \todo{(section XX Current autism spectrum disorders screening and diagnostic instruments)}, eye tracking tools can seen as a supporting diagnostic instrument, and maybe also as a supporting Level 2 screening instrument. The required basic training with the technology, and the economical investment of buying it, probably does not make eye trackers suitable to become a Level 1 screening instrument to be used at population level. At the current state of the research, it is required by the clinicians to notice at least show some sign of possible developmental disorders (among which ASD) in the children before using eye trackers as specific measuring tools. This is also due to the fact that eye tracking alone cannot provide a definitive picture of the ASD condition in small children, for a series of reasons.\\
As Minichiello (Minichiello, S., personal communication) explains, aberrant gaze behaviors are present also in other pathologies, like in severe dyspraxias or in case of motor problems which affect ocular motor mechanisms, which can be mistaken for ASD indicators. Therefore, it is fundamental to have a behavioral evaluation along with the oculomotor measurements.
Moreover, during the encounters between the children and the clinicians in an diagnostic setting, clinicians already pay close attention to children’s social and communicative gaze behavior, since it is considered an important index of possible disorders. Joint attention (triadic relation between object-parent-child), gaze following and gaze initiative, are already considered diagnostic parameters and they are already assessed on a behavioral and qualitative level. However, these constructs are probably too complex to evaluate through a controlled experimental setup with eye tracker, since ecological validity plays a fundamental role in assessing the validity of the paradigms used in the experiment (as discussed more in deep in \todo{Section XX Reliability and validity}). Eye trackers require a controlled setting in order to provide reliable and replicable measurements. On the other hand, eye trackers can provide measurements on eye movements which are impossible to assess with naked eye (e.g. saccades, smooth pursuit), and eye tracking is probably more useful in assessing these eye movements.\\
Due to these reasons eye tracking should be seen as a supporting tool for ASD diagnosis, which is meant to be used together with the already validated behavioral diagnostic praxis. In the clinical field, the technologically advanced tools should enhance the diagnosis, by integrating clinicians’ experience (mainly qualitative knowledge) with non-invasive quantitative measurements \citep{orlandi2014advancedtools}.\\
Discussing further with Simone Minichiello, it emerged that this kind of eye tracking assessments are possibly suited to be carried out during the first evaluation (anamnesis and diagnosis) and then repeated every 3 or 6 months during the clinical follow-up, basing on the type or treatment. The repetition of the assessment is functional to the observation of the speed of the improvements: the more the changes happen quickly, the less likely they are related to such a severe disorder as ASD. Small children change and develop their behaviors quickly, with evident differences emerging even monthly, especially if the children are already undergoing a somewhat heavy training (in ambulatory, at home, at school) after the clinicians suspected for the first time the presence of ASD and started a treatment program.\\
According also to his point of view, we hypothesize that—given the unobtrusiveness of eye trackers—validating this technology as a supporting tool for ASD diagnosis should not be a too hard challenge from a technical point of view. A possible first approach to start a validation process requires to:
\begin{enumerate}
    \item Define the steps of age during which making tests with the children (e.g. 18-24-36 months of age);
    \item Record a series of eye tracking measurements on a typically developing children population, in order to collect baseline data and assess the reliability of the measurements;
    \item Administer the same procedure on ASD children, in order to assess differences between groups and the internal and external validity of the method.
\end{enumerate}
A possible scenario of use of eye tracking technologies in a clinical setting is outlined by \cite{subrahmaniam2013animation}; a stand-alone equipment can be developed, acting as a psychoanalytical unit and consisting of a high-end computer with installed:
\begin{enumerate}
    \item Eye tracking hardwares and softwares;
    \item Psychometric/psychoanalytical software: built with the help, guidance and supervision of experts in psychoanalysis and psychiatry. The algorithms should compute and objectively analyze psychometrics from the eye tracking;
    \item Visual stimuli supply engine: a software which fetch from a repository of of media types and deliver visual stimuli content on-demand.
\end{enumerate}
Indeed, \cite{bocchi2012earlydiagnosis} already developed this kind of stand-alone automated solution for medical doctors or other non-technical operators, in order to collect and store on a shared server behavioral information (general movement analysis, cry analysis, behavioral analysis) on infants siblings of children diagnosed with ASD, defined to be as “high risk”. The data acquisition protocol they outline is also made for being iterated at specific time ranges. Integrating further this kind of qualitative and quantitative technology-based information collection with eye tracking could be promising in providing a more accurate picture of the children’s condition.\\
Recently, \cite{giordano2017eyetrackersystem} at University of Catania (Italy) developed a software tool which enables physicians and scientists to design and carry out procedures based on eye movement analysis for both diagnostic and research purposes, without the constant support of expert computer programmers. The system is structured to create, measure and analyze a wide variety of cognitive tests related to smooth pursuit eye movements and visually guided saccades. It does not seem to be tailored specifically on the particular kind of tasks of interest for the present study. However, the basic software and hardware infrastructure the authors describe –which is similar to the one outlined by \cite{subrahmaniam2013animation}– shows evident potential to be expanded with the framework developed in this Master Thesis. Moreover, the overall aim of their research is analogue to the one of this study, that is to overcome the complexity and the cost of creating tools for disease-specific eye tracking tests, widening the application of these technology for diagnostic purposes. Therefore, integrating the insights and preliminary results from the present study with the work from \cite{giordano2017eyetrackersystem} could be a next step in order to create a psychoanalytical unit which can assess the specific eye parameters for early detection of ASD. The authors also have already tested the system with children with ASD, but with different tasks (attention deficit test, target vs distractor fixation measurement) and with children in an older age group than the one of interest for this thesis.
