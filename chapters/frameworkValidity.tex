\section{Reliability and validity}
\label{sec:fwkreliabilityvalidity}

Using eye tracking technologies in the context of clinical evaluation of the neuropsychiatric condition of small children poses not only methodological problems, but also specific reliability and validity issues. Reliability issues involve calibration loss and the need of thorough reporting of quantitative and qualitative data about the children and their eye movements. Validity discussions are due to the choice of not assessing directly constructs like social attention (a key factor of ASD impairments) in an ecologically valid setting, but oculomotor functions in a controlled experimental setting.

The various types of reliability and validity are defined and described by \citet[pp. 89-92]{leedy2012practicalresearch} and \cite{smyrnis2008guidelines}.



\subsection{Reliability}
\label{sec:fwkreliability}

As \cite{giordano2017eyetrackersystem} summarize, eye-tracking has been applied successfully to evaluate ocular movements, especially in children/infants for whom it is usually difficult (or even impossible) to sustain attention during the medical examinations. This is possible also because current eye tracking technologies are capable to compensate effectively head movements without losing track of the eyes, providing the flexibility needed to make precise measurements without constraining the children and the setting, both in terms of procedures and equipment.

However, calibration failures have been reported in literature as a potential data loss pitfall \citep{birmingham2017gazeselection}, even though more dated or invasive eye trackers (e.g. eye tracking goggles) are more prone to this issue. Nevertheless, losing calibration during the procedure means jeopardizing both test–retest reliability and internal consistency of the data, ultimately rendering impossible to compare data within and between tests and studies. This specific issue can be particularly relevant in the context of clinical evaluation of small children. The eye tracking tests conducted by following the outlined framework are meant to be as quick as possible, due to the children’s attention span and interest, but they impose the evaluators to be flexible and prepared to take breaks and to provide the children with entertainment between the measurement trials. Therefore, after the first eye tracker calibration, it is predictable that the children would not stay still and sit in the same position throughout all the experiment, which is a potential threat to the precision of the calibration. Multiple automatic re-calibration routines in between tests and interstimulus materials are recommended \citep{sasson2012children}.

In literature, the lack of systematic reporting about important parameters, like ASD symptoms severity \citep{chitategmark2016socialattention} or the kind of training under which the participants have been exposed in their life prior to the experiments \citep{boraston2007eyetrackingASD} prevent data comparison between different studies. Therefore, in the future application of eye tracking technologies for clinical practice, not only the data related to the eye tracking measurements and analysis need to be thoroughly reported (e.g. visual angle, AOIs, size, how missing data were handled across study groups, points of calibration, number of stimuli presented, type of control comparison; \citealp{frazier2017socialgaze}), but also the information collected through anamnesis and the behavioral screening and diagnostic tests, if available.


\subsection{Validity}
\label{sec:fwkvalidity}

Some considerations about the validity of the eye tracking framework are directly related to reliability issues, while others are related to the choice of constructs to assess and their inherent validity.
Overall, given that the current study is still in an initial phase and that the eye tracking technologies are not used in clinical diagnostic settings yet, this research can show in most part theoretical validity, since the proposals done are based on literature and not on a pool of empirically collected data.

As discussed in Section~\ref{sec:earlyASD}, it is a fairly shared opinion in the HCI literature that eye tracking (especially infrared/corneal reflection types) is a relatively unobtrusive technique, which is ideal to be used on small children and which can provide insights on the different cognitive processing of stimuli and development of ASDg and TDg \citep{subrahmaniam2013animation,giordano2017eyetrackersystem,samad2017markers,bolte2016detection,falck-ytter2013eyetrackingASD}. Therefore, as a measurement technique, eye tracking shows face and content validity.

The thoroughness of the reporting of both the eye tracking and the behavioral assessments (as discussed in Section~\ref{sec:fwkreliability}) allows for the comparison and the integration between the two types of evaluations, making possible to assess if the eye tracking framework is yielding reliable results in detecting appropriate ASD related measurements. Not only this cross-reporting would make the whole framework more complete, reliable and replicable, but also and most importantly it will one day allow to compute the sensitivity, specificity and positive predictive value of this instrument (as discussed in Section~\ref{sec:screeningdiagnostics}; \citealp{charman2013measuerement}), as it is already done for behavioral screening and diagnostic instruments. These parameters will provide scientific evidence for criterion validity (both predictive and concurrent) of the eye tracking method as a supporting diagnostic tool for ASD, and therefore paving the way for the validation of this technology-based assessment on a large scale.

A concern for the validity of the framework is that it focuses more on the assessment of oculomotor functions (smooth pursuit eye movements and saccades) than the assessment of psychological constructs like attention or communication (e.g. joint attention, social interaction, interest in others, atypical eye contact) which are so far considered core symptoms and important indexes of ASD \citep{maganto2017screening}.

The construct validity of the assessment of dwelling times on social vs. non-social AOIs is based on the concept that if a child does not attend to social stimuli, he or she will not be able to develop basic cognitive functions as joint attention and theory of mind \citep{vonhofsten2009lookingevents}.

Indeed, Minichiello (Minichiello, S., personal communication) stated that it would be interesting for clinicians to have quantitative measurements about the children’s eye movements if the data focus on the social and communicative intent of the children’s gaze, rather than providing information about oculomotor disturbances, which are present in other neurological pathologies. Therefore, while the construct validity of eye movement measurements around attention or communication (e.g. fixations on social vs non-social AOIs) seem to be more immediate–even if it is not confirmed by literature, as discussed in Section~\ref{sec:fixationsstheory}– the construct validity of the assessment of basic oculomotor functions is still probably not shared enough within the clinical community. This is evidently related to the current behavioral assessment methods for ASD diagnosis.

Ecological validity is strongly related with the construct validity issue. Ecological validity can be defined as the extent to which the stimuli and protocol approximate the real-life situation that is under study \citep{boraston2007eyetrackingASD}. Tailoring the experimental setting for eye tracking methodologies means balancing between two extremes: one extreme is to have an ecologically valid setting, where eye trackers can provide few more information on eye parameters than fixations on AOIs, which by definition are arbitrarily described by the researchers as social or non-social. The other extreme is to constrain the setting for assessing more varieties of eye parameters, but making it too artificial to assess complex constructs like attention and social interactions, which need an ecologically valid setting to be investigated in a trustworthy way.

Video clips with multiple subjects interacting with each other and with objects can be considered more ecologically valid than static photographs \citep{boraston2007eyetrackingASD}, but they are all still a simulation of a real-life situation. Defining what is a socially valid content in such kind of controlled and artificial stimuli is debatable. Indeed, as \cite[ p. 185]{bush2015socialattention} state, seeing a face in the absence of any social context may not reflect how one might view that same face in more naturalistic contexts in which they typically are encountered. 

Naturalistic settings can probably be considered the most ecologically valid settings suitable for eye tracking \cite[ p. 185]{bush2015socialattention}, which poses additional processing demands like the selection of the relevant parts of the scene. As \cite{birmingham2017gazeselection} illustrate, naturalistic settings involve an experimenter who sits across from a child, engages the child’s attention by making direct eye contact with him or her, and then the experimenter looks to an peripheral object. The measure of interest is typically the percentage of trials (or pass/fail) on which the child successfully orients attention (by looking) in the direction of the experimenter’s gaze. However, the high ecological validity of these settings, with their inherent internal variability, implies to give up on precise measurements of oculomotor performance, which require consistent and strict stimuli parameters.

Another perspective is summarized from \cite{vonhofsten2009lookingevents}, in the real world social stimuli are always dynamic and embedded in a flow of events, which require rapid and accurate perception of the partner’s action and anticipation of the upcoming events in the flow of social interchange. Humans show innate attentional dispositions (e.g. to track people’s faces and to be attracted by features of faces) which allow the infants to focus their attention on the appropriate information and to enjoy interacting with other people, therefore supporting the development of social perception. ASDg show disrupted social functioning, and this can be related to different functioning of the Mirror Neuron System, preventing the children to project other people’s actions onto their own action system and therefore anticipating what is going to happen next in a social context. Literature suggests that in ASDg children the cerebellum functions differently from TDg. Cerebellum is believed to play a central role in the construction of predictive models for behavior and the anticipation of the outcome of events, which might render attention shifts slow, imprecise, and unprepared for what is going to happen next.

\cite{johnson2016review} describe that it is possible that that the widely reported anatomical abnormalities and connectivities in ASDg’s cerebellum and higher cortical areas (e.g. V5, see for example \citealp{takarae2014motionprocessing}) impact on the ability to track and predict target velocity and trajectory and positions and to integrate these signals with other actions, and these impairments are likely to contribute to a range of daily problems often described in ASDg. Smooth pursuit abnormalities could cause functional impairments in ASDg such as altered spatial awareness in environments with moving stimuli \citep{wilkes2015oculomotor}. 

As \cite{vonhofsten2009lookingevents} reflect, the social disabilities of children with ASD could be caused by some lower level dysfunctions–cognitive processes like event prediction and/or more basic perceptual process such as motion perception– which ultimately makes social interaction difficult or impossible.

\cite{takarae2004smoothpursuit} explain that sensorimotor systems are well understood both neurophysiologically and anatomically, and sensory inputs and motor responses are more easily quantifiable than most higher cognitive processes. Investigating these systems allows to collect insights into functional connectivity deficits in ASD and for determining whether some neural circuits are selectively affected by the disorder. These deficits can be responsible for impairments in higher order cognitive and adaptive behaviours. In this respect, eye tracking methods show construct and content validity for the investigation of the neural circuitry of sensorimotor systems in ASD, or possibly any neurodevelopmental disorder.

Given these reflections on construct validity in relation to social content, the framework presented in this chapter does not focus on complex constructs related to social attention. On the other hand, other constructs related to oculomotor control are assessed, which are described in literature as a possible manifestation of the different functioning of neural pathways and neurophysiology in people with ASD. Moreover, for the kind of eye parameters of interest–saccades and smooth pursuit eye movements–eye trackers can be the most reliable measurement tool, due to their high sampling frequency and precision. In this way, the eye tracking can perform its diagnosis support function without providing redundant data with the current behavioral measurements.
