\chapter*{Abstract}

Autism spectrum is a complex neurodevelopmental disorder which affects behavior, communication and social functioning. Eye tracking technologies have been becoming more viable during recent years and a noticeable interest has been put in the usage of this technology for investigating Autism Spectrum Disorders (ASD). Eye tracking can provide objective measurements of abnormalities and subtle traits in eye movements which might be potential markers of ASD. Literature suggest that eye tracking has the potential to support screening or formulating earlier and more accurate ASD diagnosis, without replacing the current established clinical practices.
The aim of the Master Thesis is to define an eye tracking framework, consisting of a rationale of relevant eye parameters and a first iteration of experimental procedure for measuring them.
An analysis of literature is compiled into the theoretical framework, in the perspective of making eye tracking technologies an effective supporting tool for clinicians. A first iteration of the framework is tested on XX typically developing children, in order to assess the overall feasibility of the application of the framework.
Results from this preliminary sperimentation show that...
\todo{instert summary of results here}

A discussion about the current state of the art and future perspectives of the implementation of eye tracking technologies is provided.
\\[1cm]

\textbf{Keywords}: autism spectrum disorders, ASD, eye tracking, eye movements, gaze pattern, eye parameters, diagnosis, assessment, children, infants.

\hypersetup{pageanchor=false}