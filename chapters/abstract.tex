\chapter*{Abstract}

Autism Spectrum Disorders (ASD) are a complex neurodevelopmental condition which affects behavior, communication and social functioning. Literature suggests that eye tracking technologies can provide objective measurements of divergent developmental patterns and subtle traits in children’s eye movements, and that these parameters can support early detection and diagnosis of ASD. An eye tracking framework is developed for children in the age group 12-24 months, and consists of a rationale of relevant eye parameters and the methodologies to assess them. A set of stimuli is programmatically designed and generated to assess smooth pursuit and saccade eye parameters. A first computerized diagnostic test procedure is designed and tested on three typically developing children (15,  24 and 9 months of age) in order to assess the overall feasibility of the application of the framework, with no aim of statistical significance and diagnostic predictive value. The results from this first experimentation suggest that the developed eye tracking framework and procedure elicit the expected gaze patterns in children belonging to the target age group, and perhaps even in younger children. Improvements need to be made on the procedure in terms of timings and quantity of recorded data. Further research is then outlined, in order to prepare the framework for the systematic collection of eye tracking data on wider samples. Systematic data collection would pave the way for future validation studies in order to support clinical diagnostic decisions.
\\[1cm]

\textbf{Keywords}: autism spectrum disorders, ASD, eye tracking, eye movements, gaze pattern, eye parameters, test design, diagnosis, assessment, children, infants.

\hypersetup{pageanchor=false}