\chapter{Introduction}
\label{chap:introduction}

The aim of this Master Thesis is to understand how eye tracking technologies can be used for the detection of Autism Spectrum Disorders (ASD). The project methodology is near to a traditional Human-Computer Interaction experimentation, starting with an in-depth literature review about the eye tracking parameters which have already been detected in previous studies as potential markers of group differences between groups of people with Autism Spectrum Disorders (hereinafter: ASDg) and Typically Developing groups (hereinafter: TDg). An eye tracking framework grounded on literature is then developed, consisting of a rationale of relevant eye parameters and guidelines for measuring them. Basing on the framework, an experimental procedure for measuring specific eye parameters about saccades and smooth pursuit eye movement on small children (12-24 months of age) is developed. Appropriate visual stimuli for eliciting the eye parameters under investigation are generated programmatically using the Processing 3 software. A first experimental protocol is developed, guiding the researchers in the arrangement of the experimental facility and equipment, and the protocol is assessed in relation to a series of research questions. Three tests with small children (9, 15 and 24 months of age) are then performed and the results of the experiments are discussed.\\

The American Psychological Association \citep{apa2017diagnosis} describes the Autism Spectrum Disorder as a complex neurodevelopmental condition that affects behavior, communication and social functioning. The Diagnostic and Statistical Manual of Mental Disorders (DSM-5) enlists the diagnostic criteria for ASD \citep{apa2013dsm5}, describing for example the presence of deficits in social-emotional reciprocity (e.g. failure of normal back-and-forth conversation), deficits in nonverbal communicative behaviors used for social interaction (e.g. abnormalities in eye contact and body language or deficits in understanding and use of gestures) and deficits in developing, maintaining, and understanding relationships (e.g. difficulties adjusting behavior to suit various social contexts). The neurodevelopmental abnormalities are present at birth and continue to evolve from the earliest months of life \citep{zwaigenbaum2005behaviorchildren}. Autism spectrum disorders are also described in the ICD-10 diagnostic manual \citep{who2016ICD10}, which also describes that the diagnosis of childhood autism (F84.0) implies “the presence of abnormal or impaired development that is manifest before the age of three years”. \\
As \cite{boraston2007eyetrackingASD} summarize, autism was first described in 1943 by the psychiatrist Leo Kanner, but in the past decade it has been suggested that autism is not a categorical disorder, but instead lies on the continuum of the autism spectrum disorders, along with Asperger syndrome and Pervasive Developmental Disorders not otherwise specified.
ASD manifestations can range from individuals with severe impairment (who may be silent, mentally disabled, and locked into repetitive behaviours) to less impaired individuals who have active but distinctly strange social approaches and communication style, and narrowly focused interests \citep{pensiero2009saccades}.\\
As \cite{wilkes2015oculomotor} describe, the oculomotor system controls volitional eye movements by incorporating visual information into appropriate motor outputs to the extraocular muscles. Therefore, assessing abnormal oculomotor performance of subjects is useful to investigate neurodevelopmental disorders, since the measurements can provide insights into aberrant neural circuitry.\\
As \cite{samad2017markers} report, currently there are no diagnostic biomarkers for ASD and, therefore, this disorder is commonly identified through direct visual observation of atypical behaviors. However,  these methods are limited in identifying subtle behavioral traits, which could be very relevant to identify specific abnormalities in behavior and social communication skills related to ASD. Automated computer vision-based tools, like eye tracking, may assist in detecting eye movement parameters. These technologies can reduce the time and expenses currently needed for screening behavioral markers in subjects with ASD and facilitate the computation of the severity and prognosis of the disorders. Associating oculomotor performance with core features of ASD can also provide suggestions for sensorimotor interventions on ASDg \citep{wilkes2015oculomotor}.\\
Current eye tracking systems capture the light reflected from eye cornea and feed raw data to the software components, in order to make heatmaps and plots (\citeauthor{subrahmaniam2013animation}, \citeyear{subrahmaniam2013animation}; \citeauthor{baxter2015understanding}, \citeyear{baxter2015understanding}, p.~437). This data provide objective and quantitative information about which parts of the scene the subject are orienting their gaze to \citep{pensiero2009saccades}, providing insights into the strategies the subjects could be using to complete tasks \citep{boraston2007eyetrackingASD}, cognitive responses to specific set of visual stimuli \citep{subrahmaniam2013animation} and provide information on the way the brain processes the visual environments at a relatively high spatial and temporal resolution \citep{vandergeest2002humanfigures}. As \cite{subrahmaniam2013animation} summarizes, eye trackers can measure several parameters of eye movements (e.g. timestamp of the gaze data, X-Y coordinates, distance of the eyes from the stimuli or display monitor, indexes of events, etc) which can be analyzed by softwares for making plots, graphs, heat maps etc. There are different kinds of eye tracking devices, each one with specific features best suited for specific experimental designs.
The Human-Computer Interaction research field, and in particular the development of technologies for diagnosis and intervention on disabilities, is strongly multidisciplinar. Cognitive sciences and software engineering are closely intertwined (\citeauthor{benyon2014designing}, \citeyear{benyon2014designing}, p.~13; \citeauthor{lazar2010researchmethods}, \citeyear{lazar2010researchmethods}, p.~17). Interaction designers are experts in developing interactive systems, both from the design and implementation perspectives. They can effectively contribute to the study of physical and cognitive impairments (ASD included), by developing practical research and assessment tools starting from reviewing scientific research and teaming up with clinicians in the field.\\
On a meta-research level, eye tracking can provide a way of communicating scientific results to non-specialist stakeholders. It is conceivable that eye tracking can be used as an integrated part of screening and diagnostic assessments in the future \citep{falck-ytter2013eyetrackingASD}. Technology-based research into ASD is more likely to drive awareness and policy changes, as it appears more rigorous and convincing to the public, and it has therefore more potential to receive media coverage  \citep{bolte2016detection}.\\

This study is not intended to be conclusive, but it is meant to provide a basis for further research about the usage of eye tracking technologies in the field of cognitive sciences and clinician-based practices.\\
At the beginning of the sections of this report, a box with a list of the key points of the section contents is provided.




\section{Eye tracking and early ASD detection}
\label{sec:earlyASD}

As \cite{bolte2016detection} summarize, eye tracking technologies have been becoming more viable during recent years and the number of published articles increased exponentially all over the world. The substantial interest in eye-tracking technology is due to the fact that this technology is currently viewed as having the highest direct clinical potential for early pediatric screening of ASD. This is possibly because it is not intrusive and can provide information on various aspects of development  \citep{bolte2016detection,falck-ytter2013eyetrackingASD,subrahmaniam2013animation,sasson2012children}.\\
\cite{sasson2012children} illustrate that eye tracking is an objective and accessible tool for examining perceptual characteristics of psychiatric disorders, which facilitates research into the abnormal visual attention and oculomotor patterns that contribute to clinical characteristics of ASD. A particularly promising application is the study on young children, in order to capture early-emerging developmental mechanisms in this critical period in the development and providing information about the early course and characteristics of ASD. Indeed eye tracking has already been largely used in studies on people with ASD (for some recent reviews, see \citealp{boraston2007eyetrackingASD,brenner2007visualsearch,falck-ytter2013eyetrackingASD,papagiannopoulou2014review,chitategmark2016socialattention,johnson2016review,bolte2016detection,frazier2017socialgaze}).\\
Diagnosing children with ASD means that, from that moment on, a rehabilitation and education path will be developed for the subjects \citep{apa2017diagnosis}. People with ASD, especially children, who had early diagnosis and intervention improve long-term prognosis \citep{vargas2016diagnosis}. Quicker and more thorough and evidence-based ASD diagnosis can lead to earlier treatment which can help the children to develop adaptation skills which allow them to attain a better level of integration into society, reducing the intensity of the condition \citep{martineau2011pupil,towie2016screening}. Not only the subjects with ASD would experience a better life experience, but also their families will improve the quality of their everyday life.\\
As \cite{anderson2006visualscanning} explain, early infancy can be a particularly advantageous time to investigate the primacy of deficits, since the deficits would presumably be less influenced by the environment and secondary deficits. For example, the preferential attention to social stimuli might be a precursor to joint attention (which seems to emerge between 6 and 12 months of age), which in turn might be a precursor to theory of mind (which seems to emerge during the fourth year of life). Differences in the neurological structures underlying a precursor could be the cause of impairments in later developing skills, therefore is important to identify early impairments in precursor skills. Various signs of diverging development in cognition and behaviors can be observed within the first 18 months of life (for a detailed description, see \citealp{shultz2015earlydepartures}). Indeed, researchers have become very interested in a downward age extension in ASD-specific detection tools that can be applied clinically \citep{towie2016screening}.\\
Analyzing specific eye movements could uncover when a child is missing valuable learning opportunities, and these kind of early indicators of atypical development are functional for research and for early interventions designed to improve social-communication in children with ASD \citep{birmingham2017gazeselection}. Impairments in ocular motor control in ASDg, such as planning, timing or accuracy of eye movements could have deep influence on visual perception, visual-motor integration, social imitation and social attention \citep{brenner2007visualsearch,johnson2016review}. Indeed, oculomotor studies provide a strategy for evaluating the functional integrity of several brain systems and cognitive processes in autism \citep{takarae2004smoothpursuit}.\\
Summarizing from \cite{kemner2004smoothpursuit} and \cite{smyrnis2008guidelines}, endophenotypic markers are measurable features (e.g. neurophysiological phenomena) which are intermediate variables measuring one aspect of the complex disorder and linking the phenotype of the disorder to the corresponding genotype. These markers can be measured in a relatively objective way and can provide insight into the underlying neurophysiology, and they are probably more associated with the biology of the disorder than clinical traits. Oculomotor function variables could serve as biomarkers of the disorder and they could be used in the evaluation and the development of treatments. So far, no neurophysiological abnormalities discovered in autism have fulfilled the criteria for being elected as an endophenotypic marker.
In this perspective, eye tracking investigations can provide a form of documentation aimed to understand if impairments in eye movements (as endophenotipic markers of different neurological functioning) have the potential to trigger the “developmental cascade” \citep{towie2016screening}  which leads the children with ASD to develop differently than TDg (as described further in \todo{Section 3.7.2, validity}). Therefore, detecting early markers of ASD can be crucial to implement treatments at earlier ages and to correct the identified causal developmental pathways \citep{young2009gazechildren}.\\
Even though literature highlights the need of early ASD diagnosis, as \cite{apa2017diagnosis} reports “Although ASD can be diagnosed as early as 15 to 18 months of age, the average age of diagnosis is about 4.5 years, and some people are not diagnosed until adulthood”. By the time most children with autism come into the clinic for an evaluation, they already have major problems with social engagement and language, as well as repetitive behaviors \citep{bourzac2012development}.
Several articles \citep{boraston2007eyetrackingASD,jones2008preferencesocial,zwaigenbaum2005behaviorchildren} point out that unobtrusive eye tracking tests done on babies could provide an effective tool for early diagnosis (even in the first 24 months), without waiting for the babies to develop further communication skills before undergoing an ASD evaluation, as well as understanding what are their opportunities for learning and development \citep{falck-ytter2013eyetrackingASD}.
Eye tracking can also provide information longitudinally \citep{zwaigenbaum2005behaviorchildren}, and it shows potential to highlight responses to intervention, such as occupational  or pharmacological therapies \citep{johnson2016review}.\\
As \cite{samad2017markers} discuss, placing electrophysiological sensors on different body parts can significantly constrain the natural body, hand, and head movements of the subjects. Many of the subjects diagnosed with ASD show anxiety and phobia related to novel experiences, therefore intrusive procedures and restrictions may overwhelm the subjects and eventually inhibit or bias their natural response data. Minimal intrusion and stress is necessary for a psychophysical studies on ASDg. In this regard, eye tracking technology shows potential in aiding early ASD diagnosis by virtue of the fact of being unobtrusive measurement technologies \citep{bolte2016detection,falck-ytter2013eyetrackingASD}, which not require electrodes positioned on eyes or head-mounted devices, record data in a relatively short time (few minutes instead of about one hour for EOG or VOG procedures) \citep{giordano2017eyetrackersystem}.\\
Even though there are many eye tracking studies done currently on ASDg, the methodologies, the paradigms and the results seems scattered and sometimes contradictory. Currently  it seems that no combination of technologies have improved the reliability or validity of diagnostic procedures or the efficacy or effectiveness of interventional practices in ASD \cite{bolte2016detection}. The framework developed in this Master Thesis (as described in detail in \todo{Section 3: Eye tracking framework}) attempts to assemble a cohesive procedure which can be applied in the clinical practice and provide more data about specific eye parameters which might prove to be related significantly with ASD when measured on large sample groups.\\
As described further in detail throughout the whole \todo{Section 2.1.: Clinical assessment}, currently the most widely used instruments for ASD detection and diagnosis are clinician-based behavioural methods. The framework developed in this Master Thesis aims to introduce a technology-based assessment for early ASD detection without completely replace traditional or gold-standard practices, acting as a supplement which benefits earlier and more accurate ASD diagnosis or even preliminary screening (as \citealp{liu2015machinelearning} also propose).\\
Summarizing, eye tracking has the potential to describe the complex picture of ASD by trying to explicitly connect the underlying neurocognitive networks to everyday function and dysfunction \citep{falck-ytter2013eyetrackingASD}, and indeed it shows significative results in distinguishing TDg and ASDg (e.g.  \citealp{boraston2007eyetrackingASD,papagiannopoulou2014review,bolte2016detection,johnson2016review}).






\section{Research questions}
\label{sec:researchquestions}

The research questions this study tries to answer, and the operationalization of them, are the following:
\begin{enumerate}
    \item Which eye tracking parameters, that so far have been investigated in literature, seem to be reliable for detecting ASD symptoms in small children (12-24 months)?
    \item What kind of experimental design (methods and stimuli) could reveal the eye tracking parameters of interest?
    \item Is the outlined procedure applicable to the target children?
    \begin{enumerate} [label=\alph*.]
        \item It was possible to conduct it until the end.
        \item The percentage of tracking ratio for each visual stimuli is >50\% \citep{sasson2012children}.
        \item The percentage of repetitions performed entirely over the total number of repetition displayed is >50\%.
        \item The diagrams of the eye tracking data show qualitatively the gaze patterns which the methods were aimed to elicit.
    \end{enumerate}
\end{enumerate}

The research questions 1 and 2 find their answer in \todo{Section 3: Eye measurement framework} . The research question 3 finds its answer in \todo{Section 5: Results of the eye tracking measurements} and \todo{Section 6: Discussion}. Given the abstract nature of question 3, an operationalization is needed. In order to consider the procedure as “applicable to the target children”, it has to:
\begin{enumerate}
    \item Be suitable to be performed from the beginning to the end (3.a), keeping the children interested
    \item Manage to provide enough eye tracking data to analyze (3.b). \cite{sasson2012children} indicate a 50\% proportion of missing data as signal of lack of overall attention to the stimuli.
    \item Provide data on a good amount of repetitions (3.c). Repetitions are important to assess the consistency of the eye movement data. If the tracking does not cover consistent parts of the recorded dataset, it would be difficult to assess the internal consistency of the data. The 50\% threshold is taken by affinity with 3.b;
    \item Elicit the expected gaze patterns, which should be similar to a reference measurement. For example, sinusoidal motion target should elicit sinusoidal eye movements.
\end{enumerate}


