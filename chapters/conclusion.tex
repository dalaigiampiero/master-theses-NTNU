\chapter{Conclusion and suggestions for further research}
\label{chap:conclusion}

The developed eye tracking framework and procedure seems to elicit the expected gaze patterns in children between 12 and 24 months of age. Currently, no computational analysis on the raw data has been performed due to the researcher’s lack of domain knowledge. The next step in this research would be to contact and team up with mathematicians and software engineers, in order to refine and complete the framework in its data analysis part. As already described in \todo{section 3.5. Data analysis}, the studies from  \cite{giordano2017eyetrackersystem,jansson2013smoothpursuit,larsson2015detection} already push the research forward in this direction, showing the interest of the software engineering community for the data analysis of complex eye movements.

If in the future it will be possible to compute precisely the eye parameters of interest from the eye tracking recordings, then it will be possible to collect data from broader samples of children in the target age. This would allow to assess systematically the construct and content validity of oculomotor performance as early ASD indicator.

If some of the parameters tested by the framework would reveal to systematically predict ASD diagnosis, a common point of developmental divergence could be found within the wide variety of symptoms and conditions of ASD. Then, investigating further the neurological basis of the oculomotor impairment comparing also the results with neuroimaging studies, would be then the natural next step for the research. It would provide a clearer picture of what are the fundamental neurological pathways involved in ASD and possibly develop treatment to overcome the divergent development. 

Contextually to the search for adequate algorithms for the analysis of the recorded eye movements, the experimental procedure needs to be refined and improved. Finding a good balance between number of repetitions necessary for having reliable results, and length of the procedure will be a priority key point to fix. Shortening and refining both stimuli and interstimulus materials is the first step.The procedure seems to elicit and measure correct gaze patterns also for children younger than the target group. However, even further refinement of the procedure is needed if it should be applied it to younger children, given their shorter attention span and the difficulty to sit still for long times

While doing more experiments on larger samples, it is also possible that less eye parameters than the ones currently implemented the framework will reveal to be more sensitive. It is also possible that multiple paradigms provide similar results and become redundant. Therefore, the number of stimuli could decrease in future developments of the framework, shortening the time needed for the whole eye tracking session and at the same time leaving more space for the most reliable paradigms.

Finding and teaming up with clinical teams and structures to conduct experiments is vital for the development of the current research. The input of these professionals will be fundamental to define and perform possible longitudinal experimental designs. These could provide insights on the children’s developmental trajectories and the sensitiveness of the eye tracking methodology over the time.
Sharing and discussing the eye tracking measurements with these experts is also of extreme importance, either them being neuropsychiatrists, logopedist, etc. Eye movement patterns could be related to behavioral manifestations apparently distant as domain.

There are a number of features in the experiment procedure and stimuli which probably need to be researched in spin-off studies and then integrated in the general framework. Indeed, the framework should act as a catalyst for further research on each micro-experimental variable.\\
The velocity of the targets should be the priority parameter to fix, since it allows to calculate the gain of the eye movement, which in literature seems to be a key divergent parameter characterizing ASDg and therefore it needs to be measured accurately.\\
Another example could be that the color of the targets in the developed stimuli is currently arbitrary. \cite{franklin2008colorASD} report less accurate color perception in children with ASD of around 11 years of age, but not when it comes to discern category of colors. If the visual variable color could impact on attention or oculomotor performance is not known at the moment. This and other micro-variables might influence the outcomes of the measurement.

The experimental stimuli have been developed to be replicable and modifiable enough to be used also in different types of eye tracking research, perhaps on subjects with other neurodevelopmental disorders as well as TDg. It is conceivable that even if the stimuli will be used for other purposes, but on children belonging to the target age group, the insights coming from other studies could help to improve the stimuli parameters for this research as well.

In conclusion, there is still a long way to go before the framework will be functional to early ASD diagnosis, both in terms of refining the procedure and harmonizing it with the current clinical practice. Literature shows noticeable interest in the subject and eye tracking is a sensible tool for investigating divergent neurodevelopmental disorders. Interaction designers can keep on contributing to the scientific debate acting as mediators, facilitators and possibly leaders within this complex multidisciplinary research field. The ultimate goal of preventing small children to develop lifelong impairments provides enough motivation itself. The key point is to make more objective and precise instruments available to clinicians, both from pragmatic and economical points of view.
