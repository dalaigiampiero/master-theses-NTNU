%\thesistitlepage % make the ordinary titlepage
\hypersetup{pageanchor=false}
%\include{summary}

\chapter*{Preface}
The following thesis is submitted as final examination for the master programme in Interaction Design at NTNU Gjøvik. The research was conducted during the spring semester of 2018.
The idea from which this research started from was about understanding more about the cognitive features of autism (ASD), a complex neurodevelopmental disorder. In particular, the interest of the researcher was focused on the processing of visual stimuli in people with ASD. This interest was driven by the intent of designing more effective Augmentative and Alternative Communication (AAC) systems in the future. However, before thinking more about how AAC systems can support the potential particular visual processing needs of people with ASD, the researcher needed to carry out more information gathering and possibly more research on how this group seems to process the information visually.
In order to narrow down the topic, which could potentially span through several theses and also several years of research, the author and the thesis supervisor, Prof. Frode Volden, agreed that a good starting point would have been to investigate the oculomotor features of people with ASD. Eye movements in humans can highlight the strategies for visual processing of stimuli and also impairments in cognitive information processing. Nowadays, these movements can be recorded accurately and non invasively by high frequency remote eye trackers, an equipment which is available at the university institution.
The author, from his master studies, already knew that experiments with eye trackers can discern between different groups of people basing on their oculomotor performance. A further example of this concept was brought by Prof. Volden, who previously studied the oculomotor differences in people with schizophrenia. Following this reasoning, by studying the oculomotor features in people with ASD, it could be possible that there are particular eye parameters which could help in ASD detection.
The author in his study and professional career had already contacts with clinicians in the ASD field and healthcare institutions, therefore he already knew more or less what are the current diagnostic procedures, and how difficult is to detect symptoms of ASD, especially in small children. Moreover, he also kenw that the current procedures for ASD diagnosis are behavior based, therefore prone to clinicians' subjective interpretation. Therefore, an objective measurement of the children's gaze patterns could improve the reliability of the current diagnostic procedures.
Summarizing, the study of the oculomotor performance of people with ASD, and in particular of children, seemed to be the first step towards various goals: from a more reliable and quick diagnosis to the design of more effective AAC systems, to the design of interfaces (both analogical and digital ones) tailored on the specific needs of people with ASD.
The early diagnosis of ASD is the first research field in which this preliminary study on oculomotor could have its application.
The author explored further this first intuition in the autumn semester studies in 2017, by conducting literature review about eye tracking parameters and methods for ASD investigations (for the IMT4898 Specialisation in Interaction Design course), and a series of interviews 
about eye tracking and ASD diagnosis (for the IMT4215 Specialization Project). From the data collected in these prevous courses, it was evident that, especially since a decade, there is a noticeable interest in the scientific community about the usage of eye tracking technologies as a support tool for early ASD diagnosis. However, the findings in literature seem to be still scattered and sometimes contradictory. For this reason, the main purpose of this master thesis is to build a framework which could put some order in the research field, by highlighting the eye tracking parameters and the procedures which show the most potential for ASD detection.
This Master Thesis report could be read by researchers interested in eye movement tracking, specific features of ASD, new approaches to clinical practice and cognitive assessment.
Readers are assumed to have a basic background in scientific research, statistical analysis and some general knowledge about ASD.\\[2cm]

%\begin{center}
%\thesiscampus, 
\thesisdate \\[1pc]
\\[1pc]
%\thesisauthor
%\end{center}